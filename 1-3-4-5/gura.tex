\documentclass[a4paper,titlepage,oneside]{article}
\usepackage[T1,T2A]{fontenc}
\usepackage[utf8]{inputenc}
\usepackage[russian]{babel}
\usepackage{amsmath}
\usepackage{graphicx}
\usepackage{soul}
\usepackage[usenames]{color}
\usepackage{colortbl}
\usepackage{listings}
\lstset{extendedchars=\true}
%\lstset{language=[LaTeX]TeX, frame=tb}
\usepackage[left = 2cm, right = 2cm, top = 2cm, bindingoffset = 0cm]{geometry}
\renewcommand{\baselinestretch}{1.65}

\title{Домашняя работа\\
 	по логическому программированию}
\author{Выполнила Гура Ярослава г. Б8403а}
\date{6 октября 2014 г.}

\begin{document}
	\maketitle

	\hspace{1em}
	
\begin{center}
    \textbf{\No1 Записать функцию эквиваленции в виде булевых формул, которые находятся в скнф и сднф.}
\end{center} 

\subsubsection*{Решение}
\begin{table}[h]
	\caption{Эквиваленция}
	\begin{center}
		\begin{tabular}{|c|c|c|c|}
			\hline
			$№$ & $A$ & $B$ & $A \longleftrightarrow B$\\
			\hline
			1 & 0 & 0 & 1 \\
			\hline
			2 & 0 & 1 & 0 \\
			\hline
			3 & 1 & 0 & 0 \\
			\hline
			4 & 1 & 1 & 1 \\
			\hline
		\end{tabular}
	\end{center}
\end{table}

\begin{quote}
    Операцию эквиваленции можно выразить через
    \begin{itemize}	
        \item отрицание ($\neg$)
        \item конъюнкцию ($\wedge$)
        \item дизъюнкцию ($\vee$)
    \end{itemize}	
    СДНФ \\
    Совершенная Дизъюктивная Нормальная Форма (СДНФ) функции - дизъюнкция полных правильных элементарных конъюнкций, равных единице на тех же наборах, что и функция. \\
    \underline{Построим СДНФ для эквиваленции}. \\
    Для каждой строки таблицы истинности, содержащей единицу, построим полную конъюнкцию (произведение аргументов).
    \begin{enumerate}
    \item Переменные, имеющие нуль в соответствующей строке, входят в произведение с отрицанием. 
    \item Переменные, имеющие значение единицы – без отрицания.
    \end{enumerate}
    Из 1-ой строки $\verb Таблицы 1$ получим: \\
        $\neg A \wedge \neg B$ \\
    Из 2-ой строки $\verb Таблицы 1$ получим: \\
        $A \wedge B$ \\
        
    Запишем дизъюнкцию конъюнкций (сумму произведений): \\
        $\neg A \wedge \neg B \vee A \wedge B$ \\ 
    Таким образом, СДНФ($A \longleftrightarrow B $) $=\ \neg A \wedge \neg B \vee A \wedge B$ \\
    
    CКНФ \\
    Совершенная Конъюктивная Нормальная Форма (СКНФ) функции - конъюнкция полных правильных элементарных дизъюнкций, равных нулю на тех же наборах, что и функция. \\
    \underline{Построим СДНФ для эквиваленции}. \\
    Для каждой строки таблицы истинности с нулевым значением функции запишем полную дизъюнкцию (логическую сумму аргументов).
    \begin{enumerate}
    \item Переменные, имеющие значения 1 в строке, входят в эту сумму с отрицанием.
    \item Переменные со значением 0 – без отрицания.
    \end{enumerate}
    Из 3-ой строки $\verb Таблицы 1$ получим: \\
        $A \vee \neg B$ \\
    Из 4-ой строки $\verb Таблицы 1$ получим: \\
        $\neg A \vee B$ \\
        
    Запишем конъюнкцию дизъюнкций (произведение сумм): \\
        $A \vee \neg B \wedge \neg A \vee B$ \\ 
    Таким образом, СКНФ($A \longleftrightarrow B $) $=\ (A \vee \neg B) \wedge (\neg A \vee B)$
\end{quote}
$\star$

\begin{center}
    \textbf{\No3 Преобразовать формулы, используя только причнно-следственные связи.}
\end{center} 

\begin{enumerate}
    \item $\forall X\ (p(X) \vee \neg q(X))$
    \item $\forall X\ (p(X) \vee \neg \exists Y(q(X,Y)\wedge r(X))$
    \item $\forall X\ (\neg p(X) \vee (q(X) \rightarrow r(X)))$
    \item $\forall X\ (r(X) \rightarrow (q(X) \rightarrow p(X)))$
\end{enumerate}

\subsubsection*{Решение}

$\textcolor{magenta}{B_{1}\& ...\& B_{n} \rightarrow A \equiv }$ \\
$\textcolor{magenta}{\equiv (\overline{B_{1} \& ...\& B_{n}}) \vee A}$ \\
$\textcolor{magenta}{\equiv \overline{B_{1}} \vee ... \vee \overline{B_{n}} \vee A }$

\begin{enumerate}
    \item $\forall X\ (p(X) \vee \neg q(X)) \equiv$ \\
          $\forall X\ (\neg q(X) \vee  p(X)) \equiv$ \\
          $\forall X\ (q(X) \rightarrow q(X)) \equiv$ \\ 
            
    
    \item $\forall X\ (p(X) \vee \textcolor{blue}{\neg \exists Y(q(X,Y)\wedge r(X))}) \textcolor{green}{\Rightarrow}$ \\
          $\textcolor{magenta}{\forall X\ \exists \neg F(X) \equiv \neg \exists X F(X)}$ \\
          $\textcolor{blue}{\neg \exists Y(q(X,Y)\wedge r(X))}\ =$ 
          $\forall X,Y\ \neg (q(X,Y) \wedge r(X))$ \\
          $\textcolor{green}{\Rightarrow} \forall X,Y\ (p(X) \vee \neg (q(X,Y) \wedge r(X))) \equiv$ \\
          $\forall X,Y\ (p(X) \vee \neg q(X,Y) \vee \neg r(X)) \equiv$ \\
          $\forall X,Y\ (\neg q(X,Y) \vee \neg r(X) \vee p(X)) \equiv$ \\ 
          $\forall X,Y\ (\overline{q}(X,Y) \vee \overline{r}(X) \vee p(X)) \equiv$ \\
          $\forall X,Y\ (q(X,Y)\ \&\ r(X) \rightarrow p(X))$           
    
    \item $\forall X\ (\neg p(X) \vee \textcolor{blue}{(q(X) \rightarrow r(X))}) \textcolor{green}{\Rightarrow}$ \\
          $\textcolor{magenta}{x \rightarrow y \equiv \neg x \vee y}$ \\
          $\textcolor{blue}{(q(X) \rightarrow r(X))}\ =\ \neg q(X) \vee r(X)$ \\
          $\textcolor{green}{\Rightarrow} \forall X\ (\neg p(X) \vee \neg q(X) \vee r(X)) \equiv$ \\
          $\forall X\ (\overline{p}(X) \vee \overline{q}(X) \vee r(X)) \equiv$ \\
          $\forall X\ (p(X)\ \&\ q(X) \rightarrow r(X))$    
    
    \item $\forall X\ (r(X) \rightarrow \textcolor{blue}{(q(X) \rightarrow p(X))}) \textcolor{green}{\Rightarrow}$ \\
          $\textcolor{magenta}{x \rightarrow y \equiv \neg x \vee y}$ \\
          $\textcolor{blue}{(q(X) \rightarrow p(X))}\ =\ \neg q(X) \vee p$ \\
          $\textcolor{green}{\Rightarrow} \forall X\ (r(X) \rightarrow (\neg q(X) \vee p(X))) \equiv$ \\
          $\forall X\ (\neg r(X) \vee (\neg q(X) \vee p(X))) \equiv$ \\
          $\forall X\ (\overline{r}(X) \vee \overline{q}(X) \vee p(X)) \equiv$ \\
          $\forall X\ (r(X)\ \&\ q(X) \rightarrow p(X))$          
\end{enumerate}
$\star$

\begin{center}
    \textbf{\No4 Fawlty Towers Hotel.}
\end{center} 

Факты:
\begin{enumerate}
\item Basil owns Fawlty Towers Hotel.
\item Basil and Sybil are married.
\item Polly and Manuel are employees at Fawlty Towers.
\item Smith and Jones are guests at Fawlty Towers.
\item Basil dislikes Manuel.
\end{enumerate}

Правила:
\begin{enumerate}
\item All hotel-owners and their spouses serve all guests at the hotel.
\item All employees at a hotel serve all guests at the hotel.
\item All employees dislike the owner of the workplace.
\end{enumerate}

$\textcolor{blue}{Find\ answers\ to\ the\ queries\ "Who\ serves\ who?"\ and\ "Who\ dislikes\ whom?".}$

\subsubsection*{Решение}
Формализуем факты: 
	\begin{center}
		\footnotesize 
		\begin{tabular}{|c|}
		    \hline own(Basil) \\
    	    \hline spouses(Basil, Sybil) \\ 
    	    \hline employee(Polly) \\
    	    \hline employee(Manuel) \\
    	    \hline guest(Smith) \\
  			\hline guest(Jones) \\
    		\hline dislik(Basil, Manuel) \\
			\hline
		\end{tabular}
		\normalsize
	\end{center}

Запишем правила: \\
All hotel-owners and their spouses serve all guests at the hotel. \\
$\forall X, Y\ serve(X, Y) \leftarrow (((\exists Z (spouses(X, Z) \wedge own(Z))) \vee own(X)) \wedge guest(Y))$ \\
$\forall X, Y\ serve(X, Y) \leftarrow \exists Z ((((spouses(X, Z) \wedge own(Z))) \vee own(X)) \wedge guest(Y))$ \\
$\textcolor{magenta}{x \rightarrow y \equiv \neg x \vee y}$ \\
$\forall X, Y\ serve(X, Y) \vee \neg \exists Z ((((spouses(X, Z) \wedge own(Z))) \vee own(X)) \wedge guest(Y))$ \\
$\textcolor{magenta}{\forall X\ \exists \neg F(X) \equiv \neg \exists X F(X)}$ \\
$\forall X, Y\ serve(X, Y) \vee \forall Z \neg ((((spouses(X, Z) \wedge own(Z))) \vee own(X)) \wedge guest(Y))$ \\
$\forall X, Y, Z\ serve(X, Y) \vee \neg ((((spouses(X, Z) \wedge own(Z))) \vee own(X)) \wedge guest(Y))$ \\
$\forall X, Y, Z\ serve(X, Y) \leftarrow ((((spouses(X, Z) \wedge own(Z))) \vee own(X)) \wedge guest(Y))$ \\
	\begin{center}
		\footnotesize 
		\begin{tabular}{|c|c|c|}

		    \hline 
		    1 &
		    $All\ hotel-owners\ and\ their\ spouses\ serve\ all\ guests\ at\ the\ hotel$ & 
		    $\forall X, Y, Z\ serve(X, Y) \leftarrow$ \\
		    & & $\leftarrow ((((spouses(X, Z) \wedge own(Z))) \vee own(X)) \wedge guest(Y))$ \\
		    
    	    \hline 
    	    2 &
    	    $All\ employees\ at\ a\ hotel\ serve\ all\ guests\ at\ the\ hotel$ &
    	    $\forall X, Y\ serve(X, Y) \leftarrow employee(X)$ \\
    	    
    	    \hline 
    	    3 &
    	    $All\ employees\ dislike\ the\ owner\ of\ the\ workplace$ &
    	    $\forall X, Y\ dislik(X, Y) \leftarrow own(X)$ \\
			\hline
		\end{tabular}
		\normalsize
	\end{center}
	
    Опираясь на факты и метоправила, можем ответить на вопросы:
    \begin{enumerate}
    \item Кто кому служит? \\
          Basil служит Smith (по первому правилу). \\
          Basil служит Jones (по первому правилу). \\
          Sybil служит Smith (по первому правилу). \\
          Sybil служит Jones (по первому правилу). \\
          Polly служит Smith (по второму правилу). \\
          Manuel служит Jones (по второму правилу). 
    \item Кто кого не любит? \\
          Polly не любит Basil (по третьему правилу). \\
          Manuel не любит Basil (по третьему правилу). \\
          Basil не любит Manuel (по факту).
    \end{enumerate}
$\star$

\begin{center}
    \textbf{\No5 Описать мир блоков.}
\end{center} 

\subsubsection*{Решение}
\begin{table}[h]
	\caption{Блоки}
	\begin{center}
		\begin{tabular}{|c|}
			\hline
			b1 \\
			\hline
			b8 \\
			\hline
			b9 \\
			\hline
			b6 \\
			\hline
			b7 \\
			\hline
		\end{tabular}
	\end{center}
\end{table}

Из таблицы блоков следует:\\
$on(b7, b6)$ \\
$on(b6, b9)$ \\
$on(b9, b8)$ \\
$on(b8, b1)$ \\

Опишем мир блоков в общей сложности:\\
Блок A находится на (on) B, a B находится на (on) C. \\  
$on(A, B).$ \\
$on(B, C).$ \\
Блок X находится выше (above) Y, если X находится на (on) Y. \\
$above(X, Y) \leftarrow on(X, Y).$ \\
Блок X находится выше (above) Y, если существует какой-то другой блок Z, размещенный на (on) Y, и X находится выше (above) Y.  \\
$above(X, Y) \leftarrow on(Z, Y),\ above(X, Z).$ \\
$\star$

\end{document}\documentclass[10pt,a4paper]{article}
\usepackage[utf8]{inputenc}
\usepackage{amsmath}
\usepackage{amsfonts}
\usepackage{amssymb}
\usepackage[left=2cm,right=2cm,top=2cm,bottom=2cm]{geometry}
\begin{document}
•
\end{document}
